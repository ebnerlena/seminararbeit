This paper analyses immersive data visualisations for high-dimensional data. It focuses on interaction methods, as this is the key part for better data exploration and thus pattern discovery. Interaction should be intuitive and effective, but this seems difficult to achieve and has not been explored much. Interaction techniques depend on the input device used, which could be mouse and keyboard on a desktop setup, touch and tangible input on a tablet, input from controllers using \ac{HMD}s, gesture input or even some individually developed techniques such as tangible markers. Interaction with a mouse is very precise as humans nowadays are used to interacting with it, but limits the user to not move around freely. On the other side gesture-based interaction is limited by the correct tracking of the hands and thus its accuracy, but is more intuitive and benefits productivity. Using \ac{HMD}s allows the user to move around freely and the interaction can be done with hand-held controllers or tangible markers. Tangible markers are effective, precise and lead to natural interaction in both physical and virtual space. In general, the interaction methods in immersive environments or 3D space can be categorised into three typical tasks: navigation, selection \& manipulation and system control. Where teleportation, \ac{OHF} or \ac{WIM} are typical for changing the perspective, selection is solved by using improved ray-casting techniques, that try to include more context of the data. Still, it is difficult to be accurate as the data points need to be selected from multiple dimensions. Finally, some open-source tools are summarized, which facilitate creating data visualisations and thus make it more broadly available without specific domain knowledge needed.